\documentclass{emulateapj} 
 
\usepackage[dvipdf]{epsfig} 
\usepackage[dvips]{rotating}
\usepackage{subfigure}
\usepackage{mathrsfs}
\usepackage{amsmath}

\newcommand{\IRAS}{{\it IRAS}}
\newcommand{\HERSCHEL}{{\it Herschel}}
\newcommand{\SPITZER}{{\it Spitzer}}
\newcommand{\PLANCK}{{\it Planck}}
\newcommand{\AKARI}{{\it AKARI}}
\newcommand{\COBE}{{\it COBE}}
\newcommand{\WISE}{{\it WISE}}

\bibpunct{(}{)}{;}{a}{}{,} 
 
\shorttitle{\PLANCK~2015 Dust Modeling}
 
\shortauthors{Meisner \& Finkbeiner} 

\begin{document}

\title{Modeling thermal dust emission in the {\it PLANCK} 2015 release maps}
\author{Aaron M. Meisner\altaffilmark{1,2}}
\author{Douglas P. Finkbeiner\altaffilmark{1,2}}
\altaffiltext{1}{Department of Physics, Harvard University, 17 Oxford Street, 
Cambridge, MA 02138, USA; ameisner@fas.harvard.edu}
\altaffiltext{2}{Harvard-Smithsonian Center for Astrophysics, 60 Garden St, 
Cambridge, MA 02138, USA; dfinkbeiner@cfa.harvard.edu}

\begin{abstract}

Based on the \PLANCK~2015 data release maps \citep{planck2015}, we present a 
thermal dust emission analysis analogous to that of \cite{meisner15}. We 
isolate thermal dust emission in the \PLANCK~frequency maps, and present 
full-sky models of the unpolarized thermal dust emission using several 
paremeterizations of the dust spectrum, including single modified blackbody and
two-component models. The qualitative conclusions of \cite{meisner15} remain 
unchanged. In particular, the zodiacal light still represents the dominant 
systematic problem of our dust model on large angular scales, and the cosmic 
infrared background anisotropy remains the dominant systematic problem on small
angular scales.

%We apply the \cite{FDS99} two-component thermal dust emission model to the
%\PLANCK~HFI maps. This parametrization of the far-infrared dust spectrum as 
%the sum of two modified blackbodies serves as an important alternative to the 
%commonly adopted single modified blackbody (MBB) dust emission model. 
%Analyzing the joint \PLANCK/DIRBE dust spectrum, we show that two-component 
%models provide a better fit to the 100-3000 GHz emission than do single-MBB 
%models, though by a lesser margin than found by \cite{FDS99} based on FIRAS 
%and DIRBE. We also derive full-sky $6.1'$ resolution maps of dust optical 
%depth and temperature by fitting the two-component model to \PLANCK~217-857 
%GHz along with  DIRBE/\IRAS~100$\mu$m data. Because our two-component model
%matches the dust spectrum near its peak, accounts for the spectrum's 
%flattening at millimeter wavelengths, and specifies dust temperature at 6.1$'$
%FWHM, our model provides reliable, high-resolution thermal dust emission 
%foreground predictions from 100 to 3000 GHz. We find that, in diffuse sky 
%regions, our two-component 100-217 GHz predictions are on average accurate to 
%within 2.2\%, while extrapolating the \cite{planckdust} single-MBB model 
%systematically underpredicts emission by 18.8\% at 100 GHz, 12.6\% at 143 GHz 
%and 7.9\% at 217 GHz. We calibrate our two-component optical depth to 
%reddening, and compare with reddening estimates based on stellar spectra. We 
%find the dominant systematic problems in our temperature/reddening maps to be 
%zodiacal light on large angular scales and the cosmic infrared background 
%anisotropy on small angular scales.

\end{abstract}

\keywords{infrared: ISM, submillimeter: ISM, dust, extinction}

\section{Introduction}
We gratefully acknowledge support from the National Science Foundation Graduate
Research Fellowship under Grant No. DGE1144152, and NASA grant NNX12AE08G. 
Based on observations obtained with Planck (http://www.esa.int/Planck), an ESA 
science mission with instruments and contributions directly funded by ESA 
Member States, NASA, and Canada. This research made use of the NASA 
Astrophysics Data System (ADS) and the IDL Astronomy User's Library at Goddard.
\footnote{Available at \texttt{http://idlastro.gsfc.nasa.gov}}

\bibliographystyle{apj}
\bibliography{twocomp}

\end{document}