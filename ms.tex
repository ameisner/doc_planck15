\documentclass{emulateapj} 
 
\usepackage[dvipdf]{epsfig} 
\usepackage[dvips]{rotating}
\usepackage{subfigure}
\usepackage{mathrsfs}
\usepackage{amsmath}
\usepackage{xcolor}

\newcommand{\IRAS}{{\it IRAS}}
\newcommand{\HERSCHEL}{{\it Herschel}}
\newcommand{\SPITZER}{{\it Spitzer}}
\newcommand{\PLANCK}{{\it Planck}}
\newcommand{\AKARI}{{\it AKARI}}
\newcommand{\COBE}{{\it COBE}}
\newcommand{\WISE}{{\it WISE}}

\bibpunct{(}{)}{;}{a}{}{,} 
 
\shorttitle{\PLANCK~2015 Dust Modeling}
 
\shortauthors{Meisner \& Finkbeiner} 

\begin{document}

\title{Modeling thermal dust emission in the {\it PLANCK} 2015 release maps}
\author{Aaron M. Meisner\altaffilmark{1,2}}
\author{Douglas P. Finkbeiner\altaffilmark{1,2}}
\altaffiltext{1}{Department of Physics, Harvard University, 17 Oxford Street, 
Cambridge, MA 02138, USA; ameisner@fas.harvard.edu}
\altaffiltext{2}{Harvard-Smithsonian Center for Astrophysics, 60 Garden St, 
Cambridge, MA 02138, USA; dfinkbeiner@cfa.harvard.edu}

\begin{abstract}

Based on the \PLANCK~2015 data release maps \citep{planck2015}, we present a 
thermal dust emission analysis analogous to that of \cite{meisner15}. We 
isolate thermal dust emission in the \PLANCK~frequency maps, and present 
full-sky models of the unpolarized thermal dust emission using several 
paremeterizations of the dust spectrum, including single modified blackbody and
two-component models. The qualitative conclusions of \cite{meisner15} remain 
unchanged. In particular, the zodiacal light still represents the dominant 
systematic problem of our dust model on large angular scales, and the cosmic 
infrared background anisotropy remains the dominant systematic problem on small
angular scales.

% mention frequency range 100-3000 GHz ?
% cite the FDS99 model in the intro as well ?
% mention that this isn't just a Planck-based SED, but includes DIRBE/IRAS ?
% mention that this is first ever thermal dust model based on Planck 2015 ?

\end{abstract}

\keywords{infrared: ISM, submillimeter: ISM, dust, extinction}

\section{Introduction}
% brief importance of dust
% iras first revolution in mapping ISM cirrus
% planck more recent breakthrough in mapping dust emission due to high
%     angular resolution / sensitivity of broad range of frequencies
% mention the existing dust data products, say they're all very useful for
%     many purposes
% but say that they all still are not perfect, and we want to investigate
%     what effect swapping in the latest planck release has on the dust model
% and we also just want to have a model with data from latest Planck release,
%     which has never been done by anyone before

% list of the sections

\section{Data}
\label{sec:data}

\section{Preprocessing}
\label{sec:prepro}

\subsection{CMB Anisotropy Removal}
\label{sec:cmb}

\subsection{Compact Sources}
\label{sec:ptsrc}

\subsection{Smoothing}
\label{sec:smth}

\subsection{Molecular Emission}
\label{sec:mole}

\subsection{Zero Level}
\label{sec:zp}

\subsubsection{Absolute Zero Level}
\label{sec:zp_abs}

\subsubsection{Relative Zero Level}
\label{sec:relzero}

\section{Dust Emission Model}
\label{sec:modeling}

\section{Predicting the Observed SED}
\label{sec:bpcorr}

\section{Global Model Parameters}
\label{sec:global}

\section{MCMC Fitting Procedure}
\label{sec:fitting}

\subsection{Pixelization}
\label{sec:pix}

\subsection{Sampling Parameters}
\label{sec:samp}

\subsection{Markov Chains}
\label{sec:mcmc}

\subsection{Low-resolution Fits}
\label{sec:lores}

\subsection{Global Parameters Revisited}
\label{sec:hier}

\section{Optical Reddening}
\label{sec:ebv}

\subsection{Reddening Calibration Procedure}
\label{sec:calib_ebv}

\subsection{Rectifying the Reddening Residuals}

\section{Comparison of Emission Predictions}
\label{sec:em_compare}

\subsection{The 353-3000 GHz Frequency Range}
\label{sec:hifreq}

\subsection{The 100-217 GHz Frequency Range}
\label{sec:lofreq}

\section{Data Release}
\label{sec:release}

\section{Conclusions}
\label{sec:conclusion}

\subsection{Single-MBB versus Two-component emission}

\subsection{Towards a Replacement for SFD}
\label{sec:replace}

We gratefully acknowledge support from the National Science Foundation Graduate
Research Fellowship under Grant No. DGE1144152, and NASA grant NNX12AE08G. 
Based on observations obtained with Planck (http://www.esa.int/Planck), an ESA 
science mission with instruments and contributions directly funded by ESA 
Member States, NASA, and Canada. This research made use of the NASA 
Astrophysics Data System (ADS) and the IDL Astronomy User's Library at Goddard.
\footnote{Available at \texttt{http://idlastro.gsfc.nasa.gov}}

\begin{center}
\begin{deluxetable*}{ccccccccc}
\tabletypesize{\scriptsize}
\tablecolumns{8} 
\tablewidth{0pc} 
\tablecaption{\label{table:offs} Input Map Properties \& Pre-processing (2015 release)} 
\tablehead{
\colhead{$\nu$ (GHz)} &
\colhead{Instrument(s)} &
\colhead{Offset ($K_{CMB}$)} & 
\colhead{Dipole ($K_{CMB}$)} & 
\colhead{$s_{857,\nu}$$\times$$u_{\nu}$} &
\colhead{$\sigma_{s_{857,\nu}}$$\times$$u_{\nu}$} &
\colhead{$n_{\nu}$} ($K_{CMB}$) &
\colhead{$c_{\nu}$} &
\colhead{FWHM ($'$)}
}
\startdata
100  & \PLANCK~HFI & \textcolor{red}{9.65$\times$10$^{-6}$$\pm$4.00$\times$10$^{-7}$} & \textcolor{red}{1.77$\times$10$^{-6}$}    & \textcolor{red}{1.63$\times$10$^{-3}$} & \textcolor{red}{3.25$\times$10$^{-5}$}  & \textcolor{red}{5.06$\times$10$^{-5}$} & \textcolor{red}{0.0009} & \textcolor{red}{9.68}   \\
143  & \PLANCK~HFI & \textcolor{red}{2.34$\times$10$^{-5}$$\pm$7.72$\times$10$^{-7}$} & \textcolor{red}{1.48$\times$10$^{-6}$}    & \textcolor{red}{4.79$\times$10$^{-3}$} & \textcolor{red}{9.58$\times$10$^{-5}$}  & \textcolor{red}{2.18$\times$10$^{-5}$} & \textcolor{red}{0.0007} & \textcolor{red}{7.30}   \\
217  & \PLANCK~HFI & \textcolor{red}{6.68$\times$10$^{-5}$$\pm$2.70$\times$10$^{-6}$} & \textcolor{red}{9.05$\times$10$^{-6}$}    & \textcolor{red}{2.17$\times$10$^{-2}$} & \textcolor{red}{4.35$\times$10$^{-4}$}  & \textcolor{red}{3.06$\times$10$^{-5}$} & \textcolor{red}{0.0016} & \textcolor{red}{5.02}   \\
353  & \PLANCK~HFI & \textcolor{red}{4.01$\times$10$^{-4}$$\pm$2.06$\times$10$^{-5}$} & \textcolor{red}{3.09$\times$10$^{-5}$}    & \textcolor{red}{9.90$\times$10$^{-2}$} & \textcolor{red}{1.98$\times$10$^{-3}$}  & \textcolor{red}{1.01$\times$10$^{-4}$} & \textcolor{red}{0.0078} & \textcolor{red}{4.94}   \\
     &             & Offset (MJy/sr)                                                  & Dipole (MJy/sr)                           & $s_{857,\nu}$                          & $\sigma_{s_{857,\nu}}$                  & $n_{\nu}$ (MJy/sr)                     &                         &          \\ \cline{3-7} \\ [-2ex]
545  & \PLANCK~HFI & \textcolor{red}{3.07$\times$10$^{-1}$$\pm$2.01$\times$10$^{-2}$} & \textcolor{red}{1.00$\times$10$^{-2}$}    & \textcolor{red}{3.36$\times$10$^{-1}$} & \textcolor{red}{6.71$\times$10$^{-3}$}  & \textcolor{red}{0.031}                 & \textcolor{red}{0.061}  & \textcolor{red}{4.83}   \\
857  & \PLANCK~HFI & \textcolor{red}{4.72$\times$10$^{-1}$$\pm$6.00$\times$10$^{-2}$} & -                                         & \textcolor{blue}{1.0}                  & \textcolor{blue}{2.0$\times$10$^{-2}$}  & \textcolor{red}{0.028}                 & \textcolor{red}{0.064}  & \textcolor{red}{4.64}   \\
1250 & DIRBE       & \textcolor{red}{7.04$\times$10$^{-2}$$\pm$1.19$\times$10$^{-1}$} & -                                         & \textcolor{red}{2.05}                  & \textcolor{red}{4.10$\times$10$^{-2}$}  & \textcolor{blue}{0.42}                 & \textcolor{blue}{0.10}  & \textcolor{blue}{42}    \\
2141 & DIRBE       & \textcolor{red}{9.94$\times$10$^{-2}$$\pm$1.53$\times$10$^{-1}$} & -                                         & \textcolor{red}{2.65}                  & \textcolor{red}{5.29$\times$10$^{-2}$}  & \textcolor{blue}{0.79}                 & \textcolor{blue}{0.10}  & \textcolor{blue}{42}    \\
3000 & DIRBE/\IRAS & \textcolor{blue}{0.0$\pm$4.3$\times$10$^{-2}$}                   & -                                         & \textcolor{red}{1.30}                  & \textcolor{red}{2.60$\times$10$^{-2}$}  & \textcolor{blue}{0.06}                 & \textcolor{blue}{0.10}  & \ \textcolor{blue}{6.1}
\enddata
\end{deluxetable*}
\end{center}

\begin{center}
\begin{deluxetable*}{llrrrrrrrrrr} 
\tabletypesize{\scriptsize}
\tablecolumns{11} 
\tablewidth{0pc} 
\tablecaption{\label{tab:global} Global Model Parameters (2015 release)} 
\tablehead{
\colhead{Number} &
\colhead{Model} &
\colhead{$f_1$} &
\colhead{$q_1/q_2$} & 
\colhead{$\beta_1$} & 
\colhead{$\beta_2$} &
\colhead{$T_2$} &
\colhead{$T_1$} &
\colhead{$n$} &
\colhead{D.O.F.} &
\colhead{$\chi^2$} &
\colhead{$\chi^2_{\nu}$}
}
\startdata
 1 & FDS99 best-fit  & 0.0363 & 13.0 & 1.67 & 2.70 & 15.67 & 9.12 & 1.055 & 7 & 55.0 & 7.86 \\
 2 & FDS99 general   & 0.0457 & 8.03 & 1.59 & 2.83 & 15.68 & 9.71 & 0.998 & 3 & 6.42 & 2.14 \\
 3 & single MBB      & 0.0    &  ... &  ... & 1.56 & 19.86 & ...  & 1.016 & 6 & 48.0 & 7.99 \\ %[-2ex]
\enddata
\end{deluxetable*}
\end{center}

\bibliographystyle{apj}
\bibliography{twocomp}

\end{document}