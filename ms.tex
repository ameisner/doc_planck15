\documentclass{emulateapj} 
 
\usepackage[dvipdf]{epsfig} 
\usepackage[dvips]{rotating}
\usepackage{subfigure}
\usepackage{mathrsfs}
\usepackage{amsmath}

\newcommand{\IRAS}{{\it IRAS}}
\newcommand{\HERSCHEL}{{\it Herschel}}
\newcommand{\SPITZER}{{\it Spitzer}}
\newcommand{\PLANCK}{{\it Planck}}
\newcommand{\AKARI}{{\it AKARI}}
\newcommand{\COBE}{{\it COBE}}
\newcommand{\WISE}{{\it WISE}}

\bibpunct{(}{)}{;}{a}{}{,} 
 
\shorttitle{\PLANCK~2015 Thermal Dust}
 
\shortauthors{Meisner \& Finkbeiner} 

\begin{document}

\title{Modeling thermal dust emission with the {\it PLANCK} 2015 release maps}
\author{Aaron M. Meisner\altaffilmark{1,2}}
\author{Douglas P. Finkbeiner\altaffilmark{1,2}}
\altaffiltext{1}{Department of Physics, Harvard University, 17 Oxford Street, 
Cambridge, MA 02138, USA; ameisner@fas.harvard.edu}
\altaffiltext{2}{Harvard-Smithsonian Center for Astrophysics, 60 Garden St, 
Cambridge, MA 02138, USA; dfinkbeiner@cfa.harvard.edu}

\begin{abstract}

Based on the \PLANCK~2015 data release maps \citep{planck2015}, we present a 
thermal dust emission analysis analogous to that of \cite{meisner15}. We 
isolate thermal dust emission in the \PLANCK~2015 frequency maps, and present 
full-sky models of the unpolarized thermal dust emission from 100 to 3000 GHz
using several paremeterizations of the dust spectrum, including two-component 
and single modified blackbody models. The qualitative conclusions of 
\cite{meisner15} remain unchanged. In particular, zodiacal light still 
represents the dominant systematic problem on large angular scales, and the 
cosmic infrared background anisotropy remains the dominant systematic problem 
on small angular scales.

% cite the FDS99 model in the intro as well ?
% mention that this isn't just a Planck-based SED, but includes DIRBE/IRAS ?
% mention that this is first ever thermal dust model based on Planck 2015 ?

\end{abstract}

\keywords{infrared: ISM, submillimeter: ISM, dust, extinction}

\section{Introduction}
% brief importance of dust
Interstellar dust in our Galaxy represents a crucial and pervasive foreground
for astronomical observations over an extremely wide range of photon energies.
In the optical, ultraviolet and near-infrared, extinction by intervening dust
attenuates and reddens the observed signal from background stars, galaxies and
transients. At millimeter wavelengths, thermal dust emission contaminates
observations of the cosmic microwave background (CMB).

% iras first revolution in mapping ISM cirrus
While the dust generally lies within a disk a few hundred parsecs in scale 
height centered about the Galactic midplane, its detailed distribution is very 
patchy and cloudlike, as first revealed in far-infrared thermal emission by the
groundbreaking \IRAS~mission \citep{low84, wheelock94}. Thus, for the sake of 
studying cosmological foregrounds and the Galactic interstellar medium (ISM), 
it is of great importance to map the distribution of dust over the entire sky, 
with the best possible angular resolution and sensitivity.

% planck more recent breakthrough in mapping dust emission due to high
%     angular resolution / sensitivity of broad range of frequencies
The \PLANCK~mission \citep{planck2013,planck2015} has recently provided a 
breakthrough in full-sky mapping of emission from interstellar dust, by 
obtaining relatively artifact free, full-sky maps in nine bandpasses from 30 
GHz to 857 GHz, with a superb combination of very high sensitivity and 
near-\IRAS~resolution from 217 GHz to 857 GHz. Although these \PLANCK~maps 
contain a superposition of various emission components, including the CMB, 
compact sources, and molecular emission, it is nevertheless possible to isolate
exquisite maps of thermal emission from interstellar dust grains.

% mention the existing dust data products, say they're all very useful for
%     many purposes
Several full-sky data products currently exist for predicting emission and/or
extinction by interstellar dust. For many years, the 
\citet[hereafter SFD]{SFD} dust map has been the industry standard for 
predicting Galactic reddening. The SFD reddening predictions rely on the
\IRAS~100$\mu$m dust emission map as a $6.1'$ FWHM template for dust optical 
depth, having modulated this emission by a $1.3^{\circ}$ FWHM temperature 
correction based on the DIRBE 140$\mu$m/240$\mu$m ratio, assuming the dust
emits as a $\beta$=2 modified blackbody (MBB). \citet[henceforth FDS99]{FDS99} 
combined the \IRAS/DIRBE data sets used by SFD with the FIRAS \citep{firas} 
dust spectra from 100-3000 GHz to extrapolate thermal dust emission from 
\IRAS~wavelengths to CMB frequencies. FDS99 found that no single MBB could
adequately fit the relatively sharp peak of the dust spectral energy 
distribution (SED) while also capturing the spectrum's flattening toward
millimeter wavelengths. FDS99 therefore fit the dust SED with a linear 
combination of two MBBs, which can be thought of as representing two 
distinct grain species in equilibrium with the same ISRF.

Both SFD/FDS99 dust temperature maps are limited to $1.3^{\circ}$ resolution 
due to their reliance on DIRBE to constrain the dust SED shape. \PLANCK's 
combination of excellent sensitivity and resolution on the Rayleigh-Jeans
side of the dust spectrum's peak allow for the construction of a full-sky dust 
temperature map with near-\IRAS~angular resolution. Multiple \PLANCK-based 
full-sky thermal dust models are presently available. \cite{planckdust} modeled
the \PLANCK-based dust SED from 353-3000 GHz using a single MBB model in which
$\beta$ is allowed to vary continuously at $0.5^{\circ}$ resolution. 
\cite{meisner15} applied the FDS99 two-component model to the \PLANCK~based
dust SED from 100-3000 GHz, finding that two MBBs provided a significantly
better fit than a single MBB over this expanded frequency range.

% but say that they all still are not perfect, and we want to investigate
%     what effect swapping in the latest planck release has on the dust model
% and we also just want to have a model with data from latest Planck release,
%     which has never been done by anyone before

Even with the enhanced temperature resolution afforded by \PLANCK, 
systematic problems with both of these recent \PLANCK~based thermal dust 
emission models still remain. Further, no thermal dust emission model yet
exists based on the \PLANCK~2015 data release. On large angular scales, 
imperfect zodical light corrections lead to the dominant systematic problem
in \PLANCK-based reddening predictions \citep{meisner15}. The cosmic infrared 
background anisotropies (CIBA) account for the majority of small-scale 
temperature and optical depth variation at high Galactic latitude. Here we 
present an update to the analysis of \cite{meisner15}, and create the first 
ever 100-3000 GHz unpolarized thermal dust emission model based on the 
\PLANCK~2015 data release.
%But none of these are yet perfect. Mention CIBA and zodi as examples from
%Meisner + Finkbeiner. Also, no Planck 2015 release thermal dust model yet
%exists at all.

% list of the sections
Much of our detailed methodology is identical to that of \cite{meisner15}. As a
result, throughout this paper, we mainly emphasize any notable differences 
relative to this previously published work, and otherwise refer the reader to 
\cite{meisner15} for thorough details.

In $\S$\ref{sec:data} we introduce the data used throughout this study. In
$\S$\ref{sec:prepro} we describe our preprocessing of the \PLANCK~maps to
isolate thermal emission from Galactic dust. In $\S$\ref{sec:modeling} we
explain the two-component emission model we apply to the \PLANCK-based dust
SED. In $\S$\ref{sec:bpcorr}, we discuss the details of predicting
\PLANCK~observations based on this dust model. In $\S$\ref{sec:global} we
derive constraints on our model's global parameters in light of the \PLANCK~HFI
maps. In $\S$\ref{sec:fitting} we detail the Markov chain Monte Carlo (MCMC)
method with which we have estimated the spatially varying parameters of our
model. In $\S$\ref{sec:ebv} we calibrate our derived optical depth to reddening
at optical wavelengths. In $\S$\ref{sec:em_compare} we compare our
two-component thermal dust emission predictions to those of \cite{planckdust}.
In $\S$\ref{sec:release} we present the full-sky maps of dust temperature and
optical depth we have obtained, and conclude in $\S$\ref{sec:conclusion}.

\section{Data}
\label{sec:data}

The \PLANCK~frequency maps utilized throughout this work are drawn from the 
\PLANCK~2015 data release \citep{planck2015}. Specifically, we have made use 
of all six of the zodiacal light corrected HFI intensity maps
\citep[\texttt{2048\_R2.00\_full\_ZodiCorrected},][]{planckzodi}.

To incorporate measurements on the Wien side of the dust emission spectrum, 
we include 100$\mu$m data in our SED fits. In particular, we use the 
\citet[henceforth SFD]{SFD} reprocessing of DIRBE/\IRAS~100$\mu$m, which we 
will refer to as \verb|i100|, and at times by frequency as 3000 GHz. The 
\verb|i100| map has angular resolution of $6.1'$, and was constructed so as to 
contain only thermal emission from Galactic dust, with compact sources and 
zodiacal light removed, and its zero level tied to H\,\textsc{i}. We use the 
\verb|i100| map as is, without any custom modifications. 

In some of our FIR dust SED analyses which do not require high angular 
resolution, specifically those of $\S$\ref{sec:global}, $\S$\ref{sec:lores}, 
and $\S$\ref{sec:hier}, we also make use of the SFD reprocessings of DIRBE 
140$\mu$m (2141 GHz) and 240$\mu$m (1250 GHz).

\section{Preprocessing}
\label{sec:prepro}

The following subsections detail the processing steps we have applied to 
isolate Galactic dust emission in the \PLANCK~maps in preparation for SED
fitting.

\subsection{CMB Anisotropy Removal}
\label{sec:cmb}

We first addressed the CMB anisotropies before performing any of the 
interpolation/smoothing described in $\S$\ref{sec:ptsrc}/$\S$\ref{sec:smth}. 
The CMB anisotropies are effectively imperceptible upon visual inspection 
of \PLANCK~857 GHz, but can be perceived at a low level in \PLANCK~545 GHz, and
are prominent at 100-353 GHz relative to the Galactic emission
we wish to characterize, especially at high latitudes. 

% mention hybrid 2013/2015 release SMICA model to account for more aggressive
% inpainting ...
To remove the CMB anisotropies, we have subtracted the Spectral Matching 
Independent Component Analysis \citep[SMICA,][]{smica} model from each of the 
\PLANCK~maps, applying appropriate unit conversions for the 545 and 857 GHz 
maps with native units of MJy/sr. Low-order corrections, particularly our 
removal of  Solar dipole residuals, are discussed in $\S$\ref{sec:zp}.

\subsection{Compact Sources}
\label{sec:ptsrc}

\subsection{Smoothing}
\label{sec:smth}

\subsection{Molecular Emission}
\label{sec:mole}

The \cite{planckdust} modeling did not attempt to correct for Galactic CO 
emission, due to the fact that CO emission is only important relative to
dust emission over a small region of the sky, predominantly in the Galactic 
plane and nearby molecular clouds. \cite{meisner15} attempted a CO correction 
based on the \cite{planckco} full-sky CO map. However, we have judged that the 
CO map becomes unreliable in the very highest-signal regions of molecular
clouds, and have therefore decided to adopt the \cite{planckdust} strategy
of making no CO correction. Because of this decision, we exclude 
CO-contaminated regions of the sky from all downstream analyses which are 
sensitive to the dust SED shape at low frequencies which are affected by CO 
line emission.

\subsection{Zero Level}
\label{sec:zp}

\subsubsection{Absolute Zero Level}
\label{sec:zp_abs}

\begin{figure}
\begin{center}
\epsfig{file=fig/zp_857.eps, width=3.4in}
\caption{\label{fig:fdsref} Scatter plot of FDS99-predicted 857 GHz thermal
dust emission versus \PLANCK~857 GHz observations, illustrating our absolute
zero level determination described in $\S$\ref{sec:zp_abs}.}
\end{center}
\end{figure}


\begin{center}
\begin{deluxetable*}{ccccccccc}
\tabletypesize{\scriptsize}
\tablecolumns{8} 
\tablewidth{0pc} 
\tablecaption{\label{table:offs} Input Map Properties \& Pre-processing (2015 release)} 
\tablehead{
\colhead{$\nu$ (GHz)} &
\colhead{Instrument(s)} &
\colhead{Offset ($K_{CMB}$)} & 
\colhead{Dipole ($K_{CMB}$)} & 
\colhead{$s_{857,\nu}$$\times$$u_{\nu}$} &
\colhead{$\sigma_{s_{857,\nu}}$$\times$$u_{\nu}$} &
\colhead{$n_{\nu}$} ($K_{CMB}$) &
\colhead{$c_{\nu}$} &
\colhead{FWHM ($'$)}
}
\startdata
100  & \PLANCK~HFI & 9.65$\times$10$^{-6}$$\pm$4.00$\times$10$^{-7}$ & 1.77$\times$10$^{-6}$    & 1.63$\times$10$^{-3}$ & 3.25$\times$10$^{-5}$  & 5.06$\times$10$^{-5}$ & 0.0009 & 9.68   \\
143  & \PLANCK~HFI & 2.34$\times$10$^{-5}$$\pm$7.72$\times$10$^{-7}$ & 1.48$\times$10$^{-6}$    & 4.79$\times$10$^{-3}$ & 9.58$\times$10$^{-5}$  & 2.18$\times$10$^{-5}$ & 0.0007 & 7.30   \\
217  & \PLANCK~HFI & 6.68$\times$10$^{-5}$$\pm$2.70$\times$10$^{-6}$ & 9.05$\times$10$^{-6}$    & 2.17$\times$10$^{-2}$ & 4.35$\times$10$^{-4}$  & 3.06$\times$10$^{-5}$ & 0.0016 & 5.02   \\
353  & \PLANCK~HFI & 4.01$\times$10$^{-4}$$\pm$2.06$\times$10$^{-5}$ & 3.09$\times$10$^{-5}$    & 9.90$\times$10$^{-2}$ & 1.98$\times$10$^{-3}$  & 1.01$\times$10$^{-4}$ & 0.0078 & 4.94   \\
     &             & Offset (MJy/sr)                                 & Dipole (MJy/sr)          & $s_{857,\nu}$         & $\sigma_{s_{857,\nu}}$ & $n_{\nu}$ (MJy/sr)    &        &        \\ \cline{3-7} \\ [-2ex]
545  & \PLANCK~HFI & 3.07$\times$10$^{-1}$$\pm$2.01$\times$10$^{-2}$ & 1.00$\times$10$^{-2}$    & 3.36$\times$10$^{-1}$ & 6.71$\times$10$^{-3}$  & 0.031                 & 0.061  & 4.83   \\
857  & \PLANCK~HFI & 4.72$\times$10$^{-1}$$\pm$6.00$\times$10$^{-2}$ & -                        & 1.0                   & 2.0$\times$10$^{-2}$   & 0.028                 & 0.064  & 4.64   \\
1250 & DIRBE       & 7.04$\times$10$^{-2}$$\pm$1.19$\times$10$^{-1}$ & -                        & 2.05                  & 4.10$\times$10$^{-2}$  & 0.42                  & 0.10   & 42     \\
2141 & DIRBE       & 9.94$\times$10$^{-2}$$\pm$1.53$\times$10$^{-1}$ & -                        & 2.65                  & 5.29$\times$10$^{-2}$  & 0.79                  & 0.10   & 42     \\
3000 & DIRBE/\IRAS & 0.0$\pm$4.3$\times$10$^{-2}$                    & -                        & 1.30                  & 2.60$\times$10$^{-2}$  & 0.06                  & 0.10   & \ 6.1
\enddata
\end{deluxetable*}
\end{center}


\subsubsection{Relative Zero Level}
\label{sec:relzero}

\section{Dust Emission Model}
\label{sec:modeling}

\begin{center}
\begin{deluxetable*}{llrrrrrrrrrr} 
\tabletypesize{\scriptsize}
\tablecolumns{11} 
\tablewidth{0pc} 
\tablecaption{\label{tab:global} Global Model Parameters (2015 release)} 
\tablehead{
\colhead{Number} &
\colhead{Model} &
\colhead{$f_1$} &
\colhead{$q_1/q_2$} & 
\colhead{$\beta_1$} & 
\colhead{$\beta_2$} &
\colhead{$T_2$} &
\colhead{$T_1$} &
\colhead{$n$} &
\colhead{D.O.F.} &
\colhead{$\chi^2$} &
\colhead{$\chi^2_{\nu}$}
}
\startdata
 1 & FDS99 best-fit  & 0.0363 & 13.0 & 1.67 & 2.70 & 15.67 & 9.12 & 1.055 & 7 & 55.0 & 7.86 \\
 2 & FDS99 general   & 0.0457 & 8.03 & 1.59 & 2.83 & 15.68 & 9.71 & 0.998 & 3 & 6.42 & 2.14 \\
 3 & single MBB      & 0.0    &  ... &  ... & 1.56 & 19.86 & ...  & 1.016 & 6 & 48.0 & 7.99 \\ %[-2ex]
\enddata
\end{deluxetable*}
\end{center}

\section{Predicting the Observed SED}
\label{sec:bpcorr}

\section{Global Model Parameters}
\label{sec:global}

\begin{figure*}
\begin{center}
\epsfig{file=fig/dirbe_slopes.eps, width=6.5in}
\caption{\label{fig:dirbe_slopes} Linear fits of SFD-reprocessed DIRBE
240$\mu$m (left), 140$\mu$m (center), and 100$\mu$m (right) as a function of
\PLANCK~857 GHz. The red lines illustrate the DIRBE correlation slopes used in
our dust emission model optimization of $\S$\ref{sec:global}.}
\end{center}
\end{figure*}

\section{MCMC Fitting Procedure}
\label{sec:fitting}

\subsection{Pixelization}
\label{sec:pix}

\subsection{Sampling Parameters}
\label{sec:samp}

\subsection{Markov Chains}
\label{sec:mcmc}

\subsection{Low-resolution Fits}
\label{sec:lores}

\subsection{Global Parameters Revisited}
\label{sec:hier}

\section{Optical Reddening}
\label{sec:ebv}

\subsection{Reddening Calibration Procedure}
\label{sec:calib_ebv}

\subsection{Rectifying the Reddening Residuals}

\section{Comparison of Emission Predictions}
\label{sec:em_compare}

\subsection{The 353-3000 GHz Frequency Range}
\label{sec:hifreq}

\subsection{The 100-217 GHz Frequency Range}
\label{sec:lofreq}

\section{Data Release}
\label{sec:release}

\begin{center}
\begin{deluxetable}{ll} 
\tabletypesize{\scriptsize}
\tablecolumns{2} 
\tablewidth{0pc} 
\tablecaption{\label{tab:bitmask} Bit-mask Summary} 
\tablehead{
\colhead{Bit} &
\colhead{Description}
}
\startdata
 0 & SFD compact source \\
 1 & SFD big object     \\
 2 & SFD no \IRAS       \\
 3 & SMICA inpainted    \\
 4 & CO emission        \\  %[-2ex]
\enddata
\end{deluxetable}
\end{center}

\begin{figure*}
\begin{center}
\epsfig{file=fig/results.eps, width=7.0in}
\caption{\label{fig:results} (top) Hot dust temperature derived from our
full-resolution two-component model fits of \PLANCK~217-857 GHz and SFD
100$\mu$m, downbinned to 27.5$'$ resolution. (bottom) Corresponding full-sky
map of best-fit two-component 545 GHz optical depth. Note that the 
prominent imprint of the zodiacal light on the dust temperature map has not
been reduced/changed significantly relative to the \cite{meisner15} 
results based on the \PLANCK~2013 data release maps.}
\end{center}
\end{figure*}

\section{Conclusions}
\label{sec:conclusion}

\subsection{Single-MBB versus Two-component emission}

\subsection{Towards a Replacement for SFD}
\label{sec:replace}

We gratefully acknowledge support from the National Science Foundation Graduate
Research Fellowship under Grant No. DGE1144152, and NASA grant NNX12AE08G. 
Based on observations obtained with Planck (http://www.esa.int/Planck), an ESA 
science mission with instruments and contributions directly funded by ESA 
Member States, NASA, and Canada. This research made use of the NASA 
Astrophysics Data System (ADS) and the IDL Astronomy User's Library at Goddard.
\footnote{Available at \texttt{http://idlastro.gsfc.nasa.gov}}

\bibliographystyle{apj}
\bibliography{twocomp}

\end{document}